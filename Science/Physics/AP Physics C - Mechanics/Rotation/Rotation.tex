\documentclass{article}
\setlength\parskip{1em plus 0.1em minus 0.2em}
\setlength\parindent{0pt}
\usepackage{ucs}
\usepackage[utf8x]{inputenc}
\everymath{\displaystyle}
\usepackage[dvipsnames]{xcolor}
\usepackage{amsfonts, amsthm, amsmath, amssymb}
\usepackage{mathtools}
\usepackage{cancel, textcomp}
\usepackage[mathscr]{euscript}
\usepackage[nointegrals]{wasysym}
\usepackage{physics}
\usepackage{tikz}
\usepackage{color}
\usepackage{microtype}
\usepackage{geometry}
\usepackage{booktabs}
\usepackage{pgfplots, pgfplotstable}
\usepackage{fancyhdr}
\usepackage{hyperref}
\usepackage{paralist}
\usepackage{float}
\usepackage{graphicx}
\usepackage{mdframed}
\usepackage{physics}
\usepackage{mathdots}
\usepackage{yhmath}
\usepackage{cancel}
\usepackage{color}
\usepackage{siunitx}
\usepackage{array}
\usepackage{multirow}
\usepackage{gensymb}
\usepackage{tabularx}
\usepackage{booktabs}
\usepackage{csquotes}
\usetikzlibrary{fadings}
\usetikzlibrary{patterns}
\usetikzlibrary{shadows.blur}
\usetikzlibrary{shapes}



\theoremstyle{definition}
    \newtheorem*{unit}{Unit}
\surroundwithmdframed[%
    backgroundcolor=teal!10!white,
    linecolor=teal,
    linewidth=2pt,
    topline=false,
    rightline=false,
    bottomline=false,
    innertopmargin=.6\baselineskip,
    innerbottommargin=.6\baselineskip]{unit}

\theoremstyle{definition}
    \newtheorem*{note}{Note}
\surroundwithmdframed[%
    backgroundcolor=pink!20!white,
    linecolor=pink,
    linewidth=2pt,
    topline=false,
    rightline=false,
    bottomline=false,
    innertopmargin=.6\baselineskip,
    innerbottommargin=.6\baselineskip]{note}

\theoremstyle{definition}
    \newtheorem*{theorem}{Theorem}
\surroundwithmdframed[%
    backgroundcolor=lime!15!white,
    linecolor=lime,
    linewidth=2pt,
    topline=false,
    rightline=false,
    bottomline=false,
    innertopmargin=.6\baselineskip,
    innerbottommargin=.6\baselineskip]{theorem}

\theoremstyle{definition}
    \newtheorem*{law}{Law}
\surroundwithmdframed[%
    backgroundcolor=lime!15!white,
    linecolor=lime,
    linewidth=2pt,
    topline=false,
    rightline=false,
    bottomline=false,
    innertopmargin=.6\baselineskip,
    innerbottommargin=.6\baselineskip]{law}

\theoremstyle{definition}
    \newtheorem*{definition}{Definition}
\surroundwithmdframed[%
    backgroundcolor=violet!10!white,
    linecolor=violet,
    linewidth=2pt,
    topline=false,
    rightline=false,
    bottomline=false,
    innertopmargin=.6\baselineskip,
    innerbottommargin=.6\baselineskip]{definition}

\theoremstyle{definition}
    \newtheorem*{example}{Example}
\surroundwithmdframed[%
    backgroundcolor=orange!10!white,
    linecolor=orange,
    linewidth=2pt,
    topline=false,
    rightline=false,
    bottomline=false,
    innertopmargin=.6\baselineskip,
    innerbottommargin=.6\baselineskip]{example}

\theoremstyle{definition}
    \newtheorem*{solution}{Solution}
\surroundwithmdframed[%
    hidealllines=true,
    innertopmargin=.6\baselineskip,
    innerbottommargin=.6\baselineskip]{solution}

\begin{document}

\begin{inprogress}
    This is done up until "Rotational Dynamics".
\end{inprogress}

\begin{note}
    These were taken from "The Princeton Review AP Physics C Prep 2021". 
\end{note}

\section{Introduction}
All motion is some combination of \textbf{translation} and \textbf{rotation}.

\section{Rotational Kinematic}
Mark several dots along a radius on a disk, and call this radius the \textit{reference line}. If the disk rotates about its center, the movement of these dots can represent angular displacement, angular velocity, and angular acceleration.

If the disk rotates as a rigid body, then all dots have the same \textbf{angular displacement}, $\Delta \theta$.
\begin{definition}
    A body is considered a \textbf{rigid body} when all points along a radial line always have the same angular displacement.
\end{definition}

\begin{definition}
    The time rate-of-change of angular displacement gives \textbf{angular velocity}, denoted by $\omega$.
    
    Average angular acceleration:
    \begin{equation*}
        \overline{\omega} = \frac{\Delta \theta}{\Delta t}
    \end{equation*}
    Instantaneous angular acceleration ($\Delta t \rightarrow 0$): 
    \begin{equation*}
        \omega = \frac{d\theta}{dt}
    \end{equation*}
\end{definition}

\begin{definition}
    The time rate-of-change of angular velocity gives \textbf{angular acceleration}, denoted by $\alpha$.

    Average angular acceleration:
    \begin{equation*}
        \overline{\alpha} = \frac{\Delta \omega}{\Delta t}
    \end{equation*}
    Instantaneous angular acceleration:
    \begin{equation*}
        \alpha = \frac{d\omega}{dt}
    \end{equation*}
\end{definition}

In the above scenario, all points undergo the same angular displacement in any given time interval; this means that all points on the disk have the same angular velocity, $\omega$, but not all points have the same linear velocity, $v$.
This follows the definition of the \textbf{radian} measure. In radians, the angular displacement, $\Delta \theta$, is related to the arc length, $\Delta s$, by the equation
\begin{equation*}
    \Delta \theta = \frac{\Delta s}{r}
\end{equation*}
Rearranging this equation and dividing by $\Delta t$
\begin{alignat*}{2}
    \Delta s = r \Delta \theta &\quad \Rightarrow \quad \frac{\Delta s}{\Delta t} = r\frac{\Delta \theta}{\Delta t} &&\quad \Rightarrow \quad \overline{v} = r \overline{\omega} \\
    ds = rd\theta &\quad \Rightarrow \quad \frac{ds}{dt} = r \frac{d\theta}{dt} &&\quad \Rightarrow \quad v=r\omega
\end{alignat*}

Therefore, the greater the value of $r$, the greater the value of $v$. Points on the rotating body farther from the rotation axis move more quickly than those closer to the rotation axis.

From the equation $v = r\omega$, one can derive the relationship that connects angular acceleration and linear acceleration. Differentiating both sides with respect to $t$ (holding $r$ constant), gives
\begin{equation*}
    \frac{dv}{dt} = r \frac{d\omega}{dt} \quad \Rightarrow \quad a=r\alpha
\end{equation*}
\begin{note}
    The acceleration $a$ in this equation is \textit{not} centripetal acceleration; it's tangential acceleration, which arises from a change in sped caused by an angular acceleration. By contrast, centripetal acceleration does not produce a change in speed.

    Often, tangential acceleration is denoted as $a_{t}$ and centripetal acceleration as $a_{c}$.
\end{note}
\begin{note}
    One can derive an expression for centripetal acceleration in terms of angular speed.
    \begin{equation*}
        a_{c} = \frac{v^{2}}{r} = \frac{(r\omega)^{2}}{r}=\omega^{2}r
    \end{equation*}
    This will be expressed on the equation sheet for the free-response section as
    \begin{equation*}
        a_{c} = \frac{v^{2}}{r}=\omega^{2}r
    \end{equation*}
\end{note}

\newpage
\section{The Big Five For Rotational Motion}

\begin{table}[!hbt]
    \begin{center}
      \setlength{\tabcolsep}{15pt}
      \renewcommand{\arraystretch}{1.5}
      \begin{tabular}{l l r}
        \multicolumn{3}{r}{\textit{Missing variable}} \\
        {Big Five \#1:} & $\Delta\theta = \overline{\omega}t$ & $\alpha$ \\
        {Big Five \#2:} & $\omega = \omega_{0} + \alpha{t}$ & $\Delta\theta$ \\
        {Big Five \#3:} & $\theta = \theta_{0} + \omega_{0}t + \frac{1}{2}\alpha{t}^{2}$ & $\omega$ \\
        {Big Five \#4:} & $\Delta\theta = \omega{t} - \frac{1}{2}\alpha(t)^{2}$ & $ \omega_{0}$ \\
        {Big Five \#5:} & $\omega^{2} = \omega_{0}^{2} + 2\alpha{\Delta}\theta$ & $t$ \\
    \end{tabular}
  \end{center} 
\end{table}

\begin{note}
    In Big Five \#1, because angular acceleration is constant, the average angular velocity is the average of the initial and the final angular velocity: $\overline{\omega} = \frac{1}{2}(\omega_{0}+\omega)$. 
    Also, if $t_{i} = 0$, then $\Delta t = t_{f} - t_{i} = t - 0 = t$, so $t$ can be written instead of $\Delta t$ in the first four equations.
\end{note}

\begin{table}[!hbt]
    \begin{center}
        \setlength{\tabcolsep}{15pt}
        \renewcommand{\arraystretch}{1.5}
        \begin{tabular}{l c c c}
            & \textit{Translational} & \textit{Rotational} & \textit{Connection}\\
            {displacement:} & $\Delta x$ & $\Delta \theta$ & $\Delta x = r \Delta \theta$ \\
            {velocity:} & $v$ & $\omega$ & $v=r\omega$ \\
            {acceleration:} & $a$ & $\alpha$ & $a=r\alpha$ \\
            {Big Five \#1:} & $\Delta x = x - x_{0} = \overline{v}t$ & $\Delta\theta = \overline{\omega}t$ & \\
            {Big Five \#2:} & $v=v_{0}+at$ & $\omega=\omega_{0}+\alpha t$ & \\
            {Big Five \#3:} & $x=x_{0}+v_{0}t+\frac{1}{2}at^{2}$ & $\Delta\theta=\omega_{0}t+\frac{1}{2}\alpha t^{2}$ & \\
            {Big Five \#4:} & $x=x_{0}+vt-\frac{1}{2}at^{2}$ & $\Delta\theta=\omega t-\frac{1}{2}\alpha t^{2}$ & \\
            {Big Five \#5:} & $v^{2}=v^{2}_{0}+2a(x-x_{0})$ & $\omega^{2}=\omega_{0}^2+2\alpha\Delta\theta$ & \\
        \end{tabular}
    \end{center}
\end{table}

\end{document}