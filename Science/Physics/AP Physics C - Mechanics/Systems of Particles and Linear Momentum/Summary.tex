\documentclass{article}
\setlength\parskip{1em plus 0.1em minus 0.2em}
\setlength\parindent{0pt}
\usepackage{ucs}
\usepackage[utf8x]{inputenc}
\everymath{\displaystyle}
\usepackage[dvipsnames]{xcolor}
\usepackage{amsfonts, amsthm, amsmath, amssymb}
\usepackage{mathtools}
\usepackage{cancel, textcomp}
\usepackage[mathscr]{euscript}
\usepackage[nointegrals]{wasysym}
\usepackage{physics}
\usepackage{tikz}
\usepackage{color}
\usepackage{microtype}
\usepackage{geometry}
\usepackage{booktabs}
\usepackage{pgfplots, pgfplotstable}
\usepackage{fancyhdr}
\usepackage{hyperref}
\usepackage{paralist}
\usepackage{float}
\usepackage{graphicx}
\usepackage{mdframed}
\usepackage{physics}
\usepackage{mathdots}
\usepackage{yhmath}
\usepackage{cancel}
\usepackage{color}
\usepackage{siunitx}
\usepackage{array}
\usepackage{multirow}
\usepackage{gensymb}
\usepackage{tabularx}
\usepackage{booktabs}
\usepackage{csquotes}
\usetikzlibrary{fadings}
\usetikzlibrary{patterns}
\usetikzlibrary{shadows.blur}
\usetikzlibrary{shapes}



\theoremstyle{definition}
    \newtheorem*{unit}{Unit}
\surroundwithmdframed[%
    backgroundcolor=teal!10!white,
    linecolor=teal,
    linewidth=2pt,
    topline=false,
    rightline=false,
    bottomline=false,
    innertopmargin=.6\baselineskip,
    innerbottommargin=.6\baselineskip]{unit}

\theoremstyle{definition}
    \newtheorem*{note}{Note}
\surroundwithmdframed[%
    backgroundcolor=pink!20!white,
    linecolor=pink,
    linewidth=2pt,
    topline=false,
    rightline=false,
    bottomline=false,
    innertopmargin=.6\baselineskip,
    innerbottommargin=.6\baselineskip]{note}

\theoremstyle{definition}
    \newtheorem*{theorem}{Theorem}
\surroundwithmdframed[%
    backgroundcolor=lime!15!white,
    linecolor=lime,
    linewidth=2pt,
    topline=false,
    rightline=false,
    bottomline=false,
    innertopmargin=.6\baselineskip,
    innerbottommargin=.6\baselineskip]{theorem}

\theoremstyle{definition}
    \newtheorem*{law}{Law}
\surroundwithmdframed[%
    backgroundcolor=lime!15!white,
    linecolor=lime,
    linewidth=2pt,
    topline=false,
    rightline=false,
    bottomline=false,
    innertopmargin=.6\baselineskip,
    innerbottommargin=.6\baselineskip]{law}

\theoremstyle{definition}
    \newtheorem*{definition}{Definition}
\surroundwithmdframed[%
    backgroundcolor=violet!10!white,
    linecolor=violet,
    linewidth=2pt,
    topline=false,
    rightline=false,
    bottomline=false,
    innertopmargin=.6\baselineskip,
    innerbottommargin=.6\baselineskip]{definition}

\theoremstyle{definition}
    \newtheorem*{example}{Example}
\surroundwithmdframed[%
    backgroundcolor=orange!10!white,
    linecolor=orange,
    linewidth=2pt,
    topline=false,
    rightline=false,
    bottomline=false,
    innertopmargin=.6\baselineskip,
    innerbottommargin=.6\baselineskip]{example}

\theoremstyle{definition}
    \newtheorem*{solution}{Solution}
\surroundwithmdframed[%
    hidealllines=true,
    innertopmargin=.6\baselineskip,
    innerbottommargin=.6\baselineskip]{solution}

\begin{document}

\begin{finished}
    This is finished.
\end{finished}

\begin{note}
    These were taken from "The Princeton Review AP Physics C Prep 2021". 
\end{note}

\section{Momentum}
\begin{itemize}
    \item Linear momentum is given by the equation $\textbf{p} = m\textbf{v}$
    \item The \textbf{Law of Conservation of Linear Momentum} states that linear momentum is conserved when no external forces act on a system.
    \begin{equation*}
        \text{total   } \textbf{p}_{\text{before collision}} = \text{total   } \textbf{p}_{\text{after collision}}
    \end{equation*}
    \item Elastic collisions \textbf{do} conserve kinetic energy.
        \begin{itemize}
            \item In general, every collision conserves energy but not necessarily kinetic energy.
        \end{itemize}
    \item Inelastic collisions \textbf{do not} conserve kinetic energy.
    \item When objects stick together, the collision is known as \textbf{perfectly inelastic}.
    \item Anytime a problem that involves collision or separation is given, first consider whether the Law of Conservation of Linear Momentum can be used.
\end{itemize}

\section{Impulse}
\begin{itemize}
    \item Impulse is given by the equation:
    \begin{equation*}
        J=\overline{F}\Delta t
    \end{equation*}
    \item The \textbf{Impulse-Momentum Theorem} states that the impulse on an object is equal to the change in momentum of the object.
    \begin{equation*}
        J=\overline{F}\Delta t = \Delta p
    \end{equation*}
\end{itemize}

\section{Center of Mass}
\begin{itemize}
    \item Usually the motion of an object is describing the motion of the center of mass. When you use Newton's Second Law, the acceleration is the acceleration of the center of mass.
    \item For point masses, $r_{\text{cm}} = \frac{\sum mr}{\sum m}$, where $r$ is used for the position of each mass.
    \item For distributed mass, the equation is:
    \begin{equation*}
        r_{\text{cm}}=\int rdm
    \end{equation*}
    \item The linear density, $\lambda dr=dm$, for $dm$ and then integrate to solve for the center of mass.
\end{itemize}

\end{document}