\documentclass{article}
\setlength\parskip{1em plus 0.1em minus 0.2em}
\setlength\parindent{0pt}
\usepackage{ucs}
\usepackage[utf8x]{inputenc}
\everymath{\displaystyle}
\usepackage[dvipsnames]{xcolor}
\usepackage{amsfonts, amsthm, amsmath, amssymb}
\usepackage{mathtools}
\usepackage{cancel, textcomp}
\usepackage[mathscr]{euscript}
\usepackage[nointegrals]{wasysym}
\usepackage{physics}
\usepackage{tikz}
\usepackage{color}
\usepackage{microtype}
\usepackage{geometry}
\usepackage{booktabs}
\usepackage{pgfplots, pgfplotstable}
\usepackage{fancyhdr}
\usepackage{hyperref}
\usepackage{paralist}
\usepackage{float}
\usepackage{graphicx}
\usepackage{mdframed}
\usepackage{physics}
\usepackage{mathdots}
\usepackage{yhmath}
\usepackage{cancel}
\usepackage{color}
\usepackage{siunitx}
\usepackage{array}
\usepackage{multirow}
\usepackage{gensymb}
\usepackage{tabularx}
\usepackage{booktabs}
\usetikzlibrary{fadings}
\usetikzlibrary{patterns}
\usetikzlibrary{shadows.blur}
\usetikzlibrary{shapes}



\theoremstyle{definition}
    \newtheorem*{unit}{Unit}
\surroundwithmdframed[%
    backgroundcolor=teal!10!white,
    linecolor=teal,
    linewidth=2pt,
    topline=false,
    rightline=false,
    bottomline=false,
    innertopmargin=.6\baselineskip,
    innerbottommargin=.6\baselineskip]{unit}

\theoremstyle{definition}
    \newtheorem*{note}{Note}
\surroundwithmdframed[%
    backgroundcolor=pink!20!white,
    linecolor=pink,
    linewidth=2pt,
    topline=false,
    rightline=false,
    bottomline=false,
    innertopmargin=.6\baselineskip,
    innerbottommargin=.6\baselineskip]{note}

\theoremstyle{definition}
    \newtheorem*{theorem}{Theorem}
\surroundwithmdframed[%
    backgroundcolor=lime!15!white,
    linecolor=lime,
    linewidth=2pt,
    topline=false,
    rightline=false,
    bottomline=false,
    innertopmargin=.6\baselineskip,
    innerbottommargin=.6\baselineskip]{theorem}

\theoremstyle{definition}
    \newtheorem*{law}{Law}
\surroundwithmdframed[%
    backgroundcolor=lime!15!white,
    linecolor=lime,
    linewidth=2pt,
    topline=false,
    rightline=false,
    bottomline=false,
    innertopmargin=.6\baselineskip,
    innerbottommargin=.6\baselineskip]{law}

\theoremstyle{definition}
    \newtheorem*{definition}{Definition}
\surroundwithmdframed[%
    backgroundcolor=violet!10!white,
    linecolor=violet,
    linewidth=2pt,
    topline=false,
    rightline=false,
    bottomline=false,
    innertopmargin=.6\baselineskip,
    innerbottommargin=.6\baselineskip]{definition}

\theoremstyle{definition}
    \newtheorem*{example}{Example}
\surroundwithmdframed[%
    backgroundcolor=orange!10!white,
    linecolor=orange,
    linewidth=2pt,
    topline=false,
    rightline=false,
    bottomline=false,
    innertopmargin=.6\baselineskip,
    innerbottommargin=.6\baselineskip]{example}

\theoremstyle{definition}
    \newtheorem*{solution}{Solution}
\surroundwithmdframed[%
    hidealllines=true,
    innertopmargin=.6\baselineskip,
    innerbottommargin=.6\baselineskip]{solution}

\theoremstyle{definition}
    \newtheorem*{inprogress}{Status}
\surroundwithmdframed[%
    backgroundcolor=yellow!20!white,
    linecolor=yellow,
    linewidth=2pt,
    topline=false,
    rightline=false,
    bottomline=false,
    innertopmargin=.6\baselineskip,
    innerbottommargin=.6\baselineskip]{inprogress}

\theoremstyle{definition}
    \newtheorem*{finished}{Status}
\surroundwithmdframed[%
    backgroundcolor=green!20!white,
    linecolor=green,
    linewidth=2pt,
    topline=false,
    rightline=false,
    bottomline=false,
    innertopmargin=.6\baselineskip,
    innerbottommargin=.6\baselineskip]{finished}

\theoremstyle{definition}
    \newtheorem*{dropped}{Status}
\surroundwithmdframed[%
    backgroundcolor=red!20!white,
    linecolor=red,
    linewidth=2pt,
    topline=false,
    rightline=false,
    bottomline=false,
    innertopmargin=.6\baselineskip,
    innerbottommargin=.6\baselineskip]{dropped}

\begin{document}

\begin{note}
    These were taken from "The Princeton Review AP Physics C Prep 2021". 
\end{note}

\section{Work}
\begin{itemize}
    \item Work is the dot product of force and displacement: $W = F \cdot x$
    \item When force varies with $x$, the work is given by the following equation: $W = \int F(x) dx$
    \item Work is positive when the force and displacement are parallel.
    \item Work is negative when the force and displacement are antiparallel.
    \item \textbf{Work-Energy Theorem}: $W = \Delta K = -\Delta U$
\end{itemize}

\section{Energy}
\begin{itemize}
    \item Kinetic energy is energy associated with motion and is given by the equation $K = \frac{1}{2}mv^{2}$
    \item Potential energy is stored energy. The potential energy due to gravity is given by the equation $U_{g} = mgh$. The potential energy due to springs is given by the equation $U_{s}=\frac{1}{2}kx^{2}$
    \item Work done by a conservative force only depends on the initial and final positions, and not on the path taken. Gravity and springs are examples of conservative forces.
    \item Work done by a non-conservative force depends on the path taken, and mechanical energy is lost by heat, sound, and so on, when these forces act on a system. Friction and air resistance are examples of non-conservative forces.
    \item \textbf{Conservation of Mechanical Energy} states that the total mechanical energy of a system is constant when there are no non-conservative forces acting on the system. It is usually written as $E_{\text{i}} = E_{\text{f}}$ or $K_{\text{i}} + U_{\text{i}} = K_{\text{f}} + U_{\text{f}}$
\end{itemize}

\section{Potential Energy Diagrams}
\begin{itemize}
    \item The potential energy can be given as $U(x)$. Then $F = -\frac{dU}{dx}$
    \item If $\frac{dU}{dx} = 0$, then $F=0$, and it is an equilibrium point.
    \item Stable equilibrium occurs when the force restores the object back toward the equilibrium point after it is disturbed.
    \item Unstable equilibrium occurs when the force moves the object further away from the equilibrium point after it is disturbed.
\end{itemize}

\section{Power}
\begin{itemize}
    \item Power is the rate at which work is done.
    \begin{flalign*}
        &P=\frac{W}{t} = \frac{dW}{dt}&&\\
        &P=Fv&&
    \end{flalign*}
\end{itemize}
\end{document}