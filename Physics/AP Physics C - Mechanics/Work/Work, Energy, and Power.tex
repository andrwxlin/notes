\documentclass{article}
\setlength\parskip{1em plus 0.1em minus 0.2em}
\setlength\parindent{0pt}
\usepackage{ucs}
\usepackage[utf8x]{inputenc}
\everymath{\displaystyle}
\usepackage[dvipsnames]{xcolor}
\usepackage{amsfonts, amsthm, amsmath, amssymb}
\usepackage{mathtools}
\usepackage{cancel, textcomp}
\usepackage[mathscr]{euscript}
\usepackage[nointegrals]{wasysym}
\usepackage{physics}
\usepackage{tikz}
\usepackage{color}
\usepackage{microtype}
\usepackage{geometry}
\usepackage{booktabs}
\usepackage{pgfplots, pgfplotstable}
\usepackage{fancyhdr}
\usepackage{hyperref}
\usepackage{paralist}
\usepackage{float}
\usepackage{graphicx}
\usepackage{mdframed}
\usepackage{physics}
\usepackage{mathdots}
\usepackage{yhmath}
\usepackage{cancel}
\usepackage{color}
\usepackage{siunitx}
\usepackage{array}
\usepackage{multirow}
\usepackage{gensymb}
\usepackage{tabularx}
\usepackage{booktabs}
\usepackage{csquotes}
\usetikzlibrary{fadings}
\usetikzlibrary{patterns}
\usetikzlibrary{shadows.blur}
\usetikzlibrary{shapes}




\theoremstyle{definition}
    \newtheorem*{unit}{Unit}
\surroundwithmdframed[%
    backgroundcolor=teal!10!white,
    linecolor=teal,
    linewidth=2pt,
    topline=false,
    rightline=false,
    bottomline=false,
    innertopmargin=0]{unit}

\theoremstyle{definition}
    \newtheorem*{note}{Note}
\surroundwithmdframed[%
    backgroundcolor=pink!20!white,
    linecolor=pink,
    linewidth=2pt,
    topline=false,
    rightline=false,
    bottomline=false,
    innertopmargin=0]{note}

\theoremstyle{definition}
    \newtheorem*{theorem}{Theorem}
\surroundwithmdframed[%
    backgroundcolor=lime!15!white,
    linecolor=lime,
    linewidth=2pt,
    topline=false,
    rightline=false,
    bottomline=false,
    innertopmargin=0]{theorem}

\theoremstyle{definition}
    \newtheorem*{definition}{Definition}
\surroundwithmdframed[%
    backgroundcolor=violet!10!white,
    linecolor=violet,
    linewidth=2pt,
    topline=false,
    rightline=false,
    bottomline=false,
    innertopmargin=0]{definition}

\theoremstyle{definition}
    \newtheorem*{example}{Example}
\surroundwithmdframed[%
    backgroundcolor=orange!10!white,
    linecolor=orange,
    linewidth=2pt,
    topline=false,
    rightline=false,
    bottomline=false,
    innertopmargin=0]{example}


\begin{document}

\noindent
\section{Introduction}
There are different forms of energy because there are different kinds of forces. 
\begin{itemize}
    \item Gravitational energy
    \item Elastic energy
    \item Thermal energy
    \item Radiant energy
    \item Electrical energy
    \item Nuclear energy
    \item Mass energy
\end{itemize}
\begin{definition}
    \textbf{Force} is an agent of change, \textbf{energy} is often defined as the ability to do work, and \textbf{work} is one way of transferring energy from one system to another.
\end{definition}
The \textbf{Law of Conservation of Energy} (aka the \textbf{First Law of Thermodynamics}) says that if you account for all its various forms, the total amount of energy in a given process will stay constant (\textit{conserved}).

\section{Work}
\begin{definition}
    Work is the energy transferred to or from an object via the application of force along a displacement.
\end{definition}
If \textbf{F} is the force, and $d \textbf{\textit{x}}$ is an infinitesimal amount of displacement, then the work done is:
\begin{equation*}
    W = \int \textbf{F} \cdot d \textbf{\textit{x}}
\end{equation*}
If \textbf{F} is constant, then it can be removed from the integral, and since $\textbf{F} \cdot \textbf{r} = (F\cos\theta)r$ for any angle $\theta$, then $W = (F\cos\theta)r$.
\begin{unit}
    The unit of work is the newton-meter (N $\cdot$ m), which is also called a joule (J).
\end{unit}
\begin{note}
    Though work depends on two vectors (\textbf{F} and \textbf{r}), work itself is \textit{not} a vector. \\
This is a result of the dot product, whereby two vectors are multiplied to produce a scalar. Another result of the dot product is that only the component of the force that is parallel (or antiparallel) to the displacement does any work.
\end{note}
\begin{note}
    Any force or component of a force that is perpendicular to the direction that an object actually moves cannot do work because the displacement in that direction is zero.
\end{note}
\begin{note}
    If a graph of force as a function of position or displacement is given, the work done by the force is the definite integral whose boundaries correspond to the displacement, $\Delta x$.
\end{note}

\section{Kinetic Energy}
\begin{definition}
    Kinetic energy, denoted by $K$, is the energy an object has by virtue of its motion.
\end{definition}
Consider an object at rest ($v_{0} = 0$) with mass $m$ and a steady force \textbf{F} is being exerted on it, pushing it in a straight line. The object's acceleration is $a = \frac{F_{\text{net}}}{m}$, so after travelling distance $\Delta x$ under the force, its final speed $v$ is:
\begin{equation*}
    v^{2} = v_{0}^2 + 2a(x-x_{0})=2a \Delta x = 2 \frac{F_{\text{net}}}{m} \Delta x \space \Rightarrow \space F_{\text{net}} \Delta x = \frac{1}{2}mv^{2}
\end{equation*}
\begin{note}
    This equation is given by the Big Five \#5.
\end{note}
But the quantity $F_{net} \Delta x$ is the \textbf{total work} done by the force, so $W_{r} = \frac{1}{2}mv^{2}$. The work done on the object has transferred energy to it, in the amount $\frac{1}{2}mv^{2}$. \\
The energy an object possesses by  virtue of its motion is therefore defined as $\frac{1}{2}mv^{2}$ and is called \textbf{kinetic energy}:
\begin{equation*}
    K = \frac{1}{2}mv^{2}
\end{equation*}
\begin{unit}
    Kinetic energy is expressed in joules.
\end{unit}
\begin{note}
    Kinetic energy is a scalar quantity.
\end{note}

\newpage
\section{The Work-Energy Theorem}
\begin{theorem}
    The total work done on an object --- equivalently, the work done by the net force --- will equal its change in kinetic energy:
    \begin{equation*}
        W_{\text{total}} = \Delta K
    \end{equation*}
    \begin{note}
        This is the extension of the derivation in Section 3 (Kinetic Energy) to an object with a non-zero initial speed.
    \end{note}
\end{theorem}

\section{Potential Energy}
\begin{definition}
    Potential energy, denoted by $U$, is the energy an object has by virtue of its position.
\end{definition}
\begin{note}
    Potential energy is independent of motion.
\end{note}
Because there are different types of forces, there are different types of potential energy.
\subsection{Gravitational Potential Energy}
\begin{definition}
    Gravitational potential energy, denoted by $U_{\text{grav}}$, is the energy stored by virtue of an object's position in a gravitational field.
\end{definition}
\begin{example}
    Consider a ball with mass $m$ being lifted from the floor to a tabletop of height $h$. The work done by gravity during the ball's ascent was:
    \begin{equation*}
        W_{\text{by gravity}} = -F_{w}h = -mgh
    \end{equation*}
\end{example}
If an object of mass $m$ is raised a height $h$ (which is small enough that $g$ stays essentially constant over this height change), then the increase in the object's $U_{\text{grav}}$ is:
\begin{equation*}
    \Delta U_{\text{grav}} = mgh
\end{equation*}
\begin{note}
    This equation does not depend on the path taken by the object. Thus, gravity is said to be a \textbf{conservative} force.
\end{note}

\newpage
\section{Conservation of Mechanical Energy}
\begin{definition}
    Mechanical energy, denoted by $E$, is the sum of an object's kinetic and potential energies:
    \begin{equation*}
        E = K + U
    \end{equation*}
\end{definition}
If the net work done by nonconservative forces is zero, the total mechanical energy of an object is conserved:
\begin{equation*}
    K_{\text{i}} + U_{\text{i}} = K_{\text{f}} + U_{\text{f}}
\end{equation*}

\section{Potential Energy Curves}
The behavior of a system can be analyzed if we are given a graph of its potential energy, $U(x)$, and its mechanical energy, $E$:
\begin{equation*}
    K+U=E \Rightarrow \frac{1}{2}mv^{2} + U(x) = E
\end{equation*}
which can be solved for $v$, the velocity at position $x$:
\begin{equation*}
    v=\pm \sqrt{\frac{2}{m}[E-U(x)]} 
\end{equation*}
Consider the following potential energy curve:

\tikzset{every picture/.style={line width=0.75pt}} %set default line width to 0.75pt        

\begin{center}

\begin{tikzpicture}[x=0.75pt,y=0.75pt,yscale=-1,xscale=1]
%uncomment if require: \path (0,310); %set diagram left start at 0, and has height of 310

%Shape: Parabola [id:dp14080636942305502] 
\draw   (200.5,64) .. controls (276.83,270.67) and (353.17,270.67) .. (429.5,64) ;
%Straight Lines [id:da9174293443598656] 
\draw    (314.5,64) -- (314.5,245) ;
%Straight Lines [id:da22088772388928146] 
\draw    (423,219) -- (207,219) ;
%Straight Lines [id:da2675552224540698] 
\draw  [dash pattern={on 4.5pt off 4.5pt}]  (235,142) -- (396.5,142) ;
%Straight Lines [id:da5512578120438989] 
\draw    (293.88,114.11) -- (305.4,134.5) ;
\draw [shift={(306.88,137.11)}, rotate = 240.52] [fill={rgb, 255:red, 0; green, 0; blue, 0 }  ][line width=0.08]  [draw opacity=0] (10.72,-5.15) -- (0,0) -- (10.72,5.15) -- (7.12,0) -- cycle    ;
%Straight Lines [id:da9813364495035721] 
\draw  [dash pattern={on 4.5pt off 4.5pt}]  (235,142) -- (235,219) ;
%Straight Lines [id:da015642540703619146] 
\draw  [dash pattern={on 4.5pt off 4.5pt}]  (396.5,142) -- (396.5,219) ;
%Shape: Circle [id:dp9251745539061926] 
\draw  [fill={rgb, 255:red, 0; green, 0; blue, 0 }  ,fill opacity=1 ] (355.2,197.65) .. controls (355.2,196.46) and (356.16,195.5) .. (357.35,195.5) .. controls (358.54,195.5) and (359.5,196.46) .. (359.5,197.65) .. controls (359.5,198.84) and (358.54,199.8) .. (357.35,199.8) .. controls (356.16,199.8) and (355.2,198.84) .. (355.2,197.65) -- cycle ;
%Straight Lines [id:da5351318033985861] 
\draw  [dash pattern={on 4.5pt off 4.5pt}]  (396.5,142) -- (396.5,219) ;
%Straight Lines [id:da11764239119148434] 
\draw  [dash pattern={on 4.5pt off 4.5pt}]  (357.5,143) -- (357.5,204) -- (357.5,220) ;
%Shape: Brace [id:dp5873443370007887] 
\draw   (363.5,198) .. controls (368.17,198) and (370.5,195.67) .. (370.5,191) -- (370.5,182) .. controls (370.5,175.33) and (372.83,172) .. (377.5,172) .. controls (372.83,172) and (370.5,168.67) .. (370.5,162)(370.5,165) -- (370.5,149) .. controls (370.5,144.33) and (368.17,142) .. (363.5,142) ;
%Shape: Brace [id:dp3456799289280761] 
\draw   (363.9,218) .. controls (366.59,218) and (367.94,216.65) .. (367.94,213.96) -- (367.94,213.96) .. controls (367.94,210.12) and (369.28,208.2) .. (371.97,208.2) .. controls (369.28,208.2) and (367.94,206.28) .. (367.94,202.44)(367.94,204.16) -- (367.94,202.44) .. controls (367.94,199.75) and (366.59,198.4) .. (363.9,198.4) ;
%Straight Lines [id:da5674445004043329] 
\draw    (427.5,141.43) -- (384.01,169.79) ;
\draw [shift={(381.5,171.43)}, rotate = 326.89] [fill={rgb, 255:red, 0; green, 0; blue, 0 }  ][line width=0.08]  [draw opacity=0] (10.72,-5.15) -- (0,0) -- (10.72,5.15) -- (7.12,0) -- cycle    ;
%Straight Lines [id:da9886442824126032] 
\draw    (427.5,173.43) -- (379,205.76) ;
\draw [shift={(376.5,207.43)}, rotate = 326.31] [fill={rgb, 255:red, 0; green, 0; blue, 0 }  ][line width=0.08]  [draw opacity=0] (10.72,-5.15) -- (0,0) -- (10.72,5.15) -- (7.12,0) -- cycle    ;

% Text Node
\draw (323,58) node [anchor=north west][inner sep=0.75pt]   [align=left] {$E$};
% Text Node
\draw (251,93) node [anchor=north west][inner sep=0.75pt]   [align=left] {$E=E_{0}$};
% Text Node
\draw (223.5,224) node [anchor=north west][inner sep=0.75pt]   [align=left] {$\mbox{-}x_{0}$};
% Text Node
\draw (386.5,222) node [anchor=north west][inner sep=0.75pt]   [align=left] {$x_{0}$};
% Text Node
\draw (430.5,214) node [anchor=north west][inner sep=0.75pt]   [align=left] {$x$};
% Text Node
\draw (431.5,133) node [anchor=north west][inner sep=0.75pt]   [align=left] {$K$};
% Text Node
\draw (431.5,165) node [anchor=north west][inner sep=0.75pt]   [align=left] {$U$};

\end{tikzpicture}

\end{center}
The graph shows how $U$ varies with $x$. A particular value of the total energy, $E = E_{0}$, is also shown.
Motion of an object whose potential energy is given by $U(x)$ and which has a mechanical energy of $E_{0}$ is confined to the region $-x_{0} \le x \le x_{0}$, because only in this range is $E_{0} \ge U(x)$.
At each $x$ in this range, the kinetic energy $K = E_{0} - U(x)$ is positive. however, if $x > x_{0}$ (or if $x < -x_{0}$), then $U(x) > E_{0}$, which is physically impossible because the difference, $E_{0} - U(x)$, which should give $K$, is negative.

\end{document}