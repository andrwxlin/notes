\documentclass{article}
\setlength\parskip{1em plus 0.1em minus 0.2em}
\setlength\parindent{0pt}
\usepackage{ucs}
\usepackage[utf8x]{inputenc}
\everymath{\displaystyle}
\usepackage[dvipsnames]{xcolor}
\usepackage{amsfonts, amsthm, amsmath, amssymb}
\usepackage{mathtools}
\usepackage{cancel, textcomp}
\usepackage[mathscr]{euscript}
\usepackage[nointegrals]{wasysym}
\usepackage{physics}
\usepackage{tikz}
\usepackage{color}
\usepackage{microtype}
\usepackage{geometry}
\usepackage{booktabs}
\usepackage{pgfplots, pgfplotstable}
\usepackage{fancyhdr}
\usepackage{hyperref}
\usepackage{paralist}
\usepackage{float}
\usepackage{graphicx}
\usepackage{mdframed}
\usepackage{physics}
\usepackage{mathdots}
\usepackage{yhmath}
\usepackage{cancel}
\usepackage{color}
\usepackage{siunitx}
\usepackage{array}
\usepackage{multirow}
\usepackage{gensymb}
\usepackage{tabularx}
\usepackage{booktabs}
\usepackage{csquotes}
\usetikzlibrary{fadings}
\usetikzlibrary{patterns}
\usetikzlibrary{shadows.blur}
\usetikzlibrary{shapes}




\theoremstyle{definition}
    \newtheorem*{unit}{Unit}
\surroundwithmdframed[%
    backgroundcolor=teal!10!white,
    linecolor=teal,
    linewidth=2pt,
    topline=false,
    rightline=false,
    bottomline=false,
    innertopmargin=0]{unit}

\theoremstyle{definition}
    \newtheorem*{note}{Note}
\surroundwithmdframed[%
    backgroundcolor=pink!20!white,
    linecolor=pink,
    linewidth=2pt,
    topline=false,
    rightline=false,
    bottomline=false,
    innertopmargin=0]{note}

\begin{document}

\noindent
\section{Introduction}
There are different forms of energy because there are different kinds of forces. 
\begin{itemize}
    \item Gravitational energy
    \item Elastic energy
    \item Thermal energy
    \item Radiant energy
    \item Electrical energy
    \item Nuclear energy
    \item Mass energy
\end{itemize}
\textbf{Force} is an agent of change, \textbf{energy} is often defined as the ability to do work, and \textbf{work} is one way of transferring energy from one system to another.
The \textbf{Law of Conservation of Energy} (aka the \textbf{First Law of Thermodynamics}) says that if you account for all its various forms, the total amount of energy in a given process will stay constant (\textit{conserved}).

\section{Work}
If \textbf{F} is the force, and $d \textbf{\textit{x}}$ is an infinitesimal amount of displacement, then the work done is:
\begin{equation*}
    W = \int \textbf{F} \cdot d \textbf{\textit{x}}
\end{equation*}
If \textbf{F} is constant, then it can be removed from the integral, and since $\textbf{F} \cdot \textbf{r} = (F\cos\theta)r$ for any angle $\theta$, then $W = (F\cos\theta)r$.
\begin{unit}
    The unit of work is the newton-meter (N $\cdot$ m), which is also called a joule (J).
\end{unit}
\begin{note}
    Though work depends on two vectors (\textbf{F} and \textbf{r}), work itself is \textit{not} a vector. \\
This is a result of the dot product, whereby two vectors are multiplied to produce a scalar. Another result of the dot product is that only the component of the force that is parallel (or antiparallel) to the displacement does any work.
\end{note}
\begin{note}
    Any force or component of a force that is perpendicular to the direction that an object actually moves cannot do work because the displacement in that direction is zero.
\end{note}
\begin{note}
    If a graph of force as a function of position or displacement is given, the work done by the force is the definite integral whose boundaries correspond to the displacement, $\Delta x$.
\end{note}

\noindent
\section{Kinetic Energy}
Consider an object at rest ($v_{0} = 0$) with mass $m$ and a steady force \textbf{F} is being exerted on it, pushing it in a straight line. The object's acceleration is $a = \frac{F_{\text{net}}}{m}$, so after travelling distance $\Delta x$ under the force, its final speed $v$ is:
\begin{equation*}
    v^{2} = v_{0}^2 + 2a(x-x_{0})=2a \Delta x = 2 \frac{F_{\text{net}}}{m} \Delta x \rightarrow F_{\text{net}} \Delta x = \frac{1}{2}mv^{2}
\end{equation*}
\begin{note}
    This equation is given by the Big Five \#5.
\end{note}
But the quantity $F_{net} \Delta x$ is the \textbf{total work} done by the force, so $W_{r} = \frac{1}{2}mv^{2}$. The work done on the object has transferred energy to it, in the amount $\frac{1}{2}mv^{2}$. \\
The energy an object possesses by  virtue of its motion is therefore defined as $\frac{1}{2}mv^{2}$ and is called \textbf{kinetic energy}:
\begin{equation*}
    K = \frac{1}{2}mv^{2}
\end{equation*}
\end{document}